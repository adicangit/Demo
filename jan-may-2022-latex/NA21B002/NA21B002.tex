\documentclass{article}
\title{\textbf{Euler's identity}}
\date{June 2022}
\author{Aditya Chopra NA21B002}

\usepackage{graphicx}


\begin{document}
  \maketitle
\large
\paragraph{}\textbf{Euler's identity} is considered to be an exemplar of mathematical beauty as it shows a profound connection between the most fundamental numbers in mathematics. In addition, it is directly used in a proof that $\pi$ is not algebraic ie. not the root of a non-zero polynomial of finite degree with rational coefficients, which implies the impossibility of squaring the circle ie. It is not possible to construct a square with the area of a circle by using only a finite number of steps with a compass and straightedge.

\vspace{1cm}


\boldmath
\begin{equation}
  e^{i\pi}+1=0
\end{equation}

\vspace{1cm}

Euler's identity is named after the Swiss mathematician Leonhard Euler. It is a special case of Euler's formula,

${\displaystyle e^{ix}=\cos x+i\sin x}$ when evaluated for x = $\pi$

\vspace{1cm}

\begin{tabular}{|c|l|}
    \hline
    $e$ & Euler's number, the base of natural logarithms\\
    $i$ & The imaginary unit, which by definition satisfies $i$^{2}=     -1\\
    $\pi$ & The ratio of the circumference of a circle to its diameter\\
    \hline
\end{tabular}


\end{document}
